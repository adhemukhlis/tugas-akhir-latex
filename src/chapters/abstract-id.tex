\clearpage
\chapter*{ABSTRAK}
\addcontentsline{toc}{chapter}{ABSTRAK}
\begin{center}
\MakeTextUppercase{\textbf{\large{\thetitle}}}

Oleh

\MakeTextUppercase{\theauthor}
\end{center}
\medskip
\begin{spacing}{1.0}
Penggunaan \textit{spreadsheet} di dalam kehidupan sehari-hari tidak terlepas dari pengumpulan data. Hal ini disebabkan oleh mudahnya penggunaan \textit{spreadsheet} sehingga banyak orang awam yang memilih menggunakan \textit{spreadsheet} dibandingkan basis data. Pengumpulan data menggunakan aplikasi \textit{spreadsheet} memiliki beberapa kelemahan seperti tidak adanya validasi, terisolasinya data yang dikumpulkan, serta terdapat kemungkinan sulitnya berkolaborasi didalam pengumpulan data. Sehingga masalah yang ingin diselesaikan disini adalah transformasi data menjadi bentuk basis data, melakukan verifikasi terhadap data, dan dapat dilakukan secara kolaboratif.

Pada laporan ini akan dibahas mengenai cara mentransformasikan data yang terdapat pada \textit{spreadsheet} sehingga dapat diolah ke dalam bentuk basis data. Transformasi ini akan terdiri dari 4 tahap utama yakni, \textit{clustering}, \textit{row identification}, \textit{cell identification}, dan \textit{header-data assignment}. Algoritma yang akan digunakan adalah Hierarchical Clustering serta Conditional Random Field, data pembelajaran didapatkan dari Statistical Abstract of the United States (SAUS) 2010 serta pengumpulan manual. Validasi akan dilakukan terhadap 3 tipe validasi yakni, tipe data, domain data, dan relasi antar data. Algoritma tersebut akan dibangun diatas \textit{spreadsheet} kolaboratif bernama Ethercalc sebagai solusi untuk dapat melakukan pengumpulan data secara kolaboratif.

Diakhir laporan akan dibahas mengenai hasil percobaan menggunakan perangkat lunak yang telah dibangun. Pengujian dilakukan dengan menggunakan beberapa data set yang dibuat dan diujikan kebenaran hasil algoritma serta kemampuannya untuk mengintrepertasikan data masukan menjadi data pada basis data sehingga didapatkan tingkat akurasi dari perangkat lunak yang dibangun.

Kata kunci: \textit{spreadsheet}, pengumpulan data, \textit{data quality}, \textit{data management}.
\end{spacing}

\clearpage