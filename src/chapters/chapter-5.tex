\chapter{Kesimpulan dan Saran}
Bab ini berisi hal-hal yang dapat disimpulkan dari pelaksanaan Tugas Akhir ini. Bab ini juga mencakup saran untuk pengembangan Tugas Akhir ini di masa mendatang.

\section{Kesimpulan}
Berdasarkan hasil pengembangan kakas pengumpulan data menggunakan \textit{spreadsheet} yang telah dilakukan. Berikut adalah kesimpulan yang diperoleh.
\begin{enumerate}
	\item Telah berhasil melakukan penambahan fitur pada aplikasi EtherCalc yang diharapkan dapat mempermudah pengumpulan data ke basis data.
	\item Konflik pada kolaborasi dapat ditangani dengan baik oleh sistem pada EtherCalc sehingga penanganan konflik tidak perlu dibuat kembali.
	\item Data yang akan dimasukkan ke basis data penyimpanan berhasil di validasi menggunakan fitur yang dibuat dengan tiga tipe validasi yakni tipe data, domain data, dan relasi data.
	\item Identifikasi tabel pada suatu \textit{sheet} dapat dilakukan dengan menggunakan algoritma hierarchical clustering. Identifikasi label suatu baris pada tabel dapat dilakukan dengan teknik framefinder dengan membagi label menjadi empat jenis yakni \textit{title}, \textit{data}, \textit{header}, dan \textit{footer}. Jika teknik framefinder tidak berhasil menemukan label dan data, maka pengguna dapat memasukkan konfigurasi tabel secara manual dan mengubahnya sesuai dengan keinginan pengguna. 
	\item Penggabungan data antar \textit{spreadsheet} dapat dilakukan dengan fitur yang dibuat dan dapat digabungkan secara horizontal, vertikal, maupun gabungan keduanya. Data-data pada \textit{spreadsheet} berhasil dimasukkan ke dalam basis data yang ditentukan sesuai dengan \textit{metadata table} yang telah dibuat pengguna maupun hasil pencarian otomatis dari algoritma framefinder.
	\item Alur kerja pengumpulan data berubah sehingga tidak lagi diperlukan pengumpulan \textit{spreadsheet} secara lokal dan melakukan \textit{versioning} karena seluruh data berada pada satu tempat menggunakan mekanisme penyimpanan oleh EtherCalc. Pada saat pengumpulan data dari berbagai \textit{spreadsheet} dapat dilakukan menggunakan platform yang sama yakni EtherCalc, sehingga tidak memerlukan bantuan aplikasi lain ataupun manual. Hasil akhir dari pengumpulan data merupakan data pada basis data sehingga data mudah diolah, ditampilkan, maupun diubah menggunakan banyak aplikasi yang tersedia.
\end{enumerate}

\section{Saran}
Saran yang dapat diberikan untuk pengembangan di masa mendatang adalah sebagai berikut:
\begin{enumerate}
	\item Pada pembangunan selanjutnya dapat ditambahkan fitur untuk menangani kasus tabel yang lebih rumit seperti formulir.
	\item Penambahan data pembelajaran untuk identifikasi label baris dapat dilakukan sehingga akan memperbaiki hasil identifikasi otomatis. Pada pengembangan selanjutnya dapat ditambahkan \textit{feedback} dari pengguna sebagai data pembelajaran.
	\item Menambahkan fungsionalitas yakni memperbolehkan kolom \textit{key} lebih dari satu pada \textit{metadata table}.
	\item Menambahkan jenis validasi contohnya validasi formula.
\end{enumerate}