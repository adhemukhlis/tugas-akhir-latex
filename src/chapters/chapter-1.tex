\chapter{Pendahuluan}

Pada bab ini akan dibahas mengenai gambaran dasar dari pelaksanaan tugas akhir dalam bentuk penjelasan latar belakang yang mendasari pemilihan topik. Dari latar belakang tersebut, akan diurai kembali menjadi rumusan masalah, tujuan, batasan masalah, serta metodologi yang digunakan untuk keperluan Tugas Akhir ini.

\section{Latar Belakang}

Aplikasi \textit{spreadsheet} menjadi aplikasi yang mudah ditemui dan wajar digunakan oleh banyak orang secara personal maupun dalam sebuah organisasi komersial \parencite{Chan1996}. Pada tahun 1979, aplikasi \textit{spreadsheet} pertama dibuat dengan nama VisiCalc. Pengguna komputer pada saat itu dimanjakan dengan kapabilitas dan fleksibilitas aplikasi yang dapat melakukan operasi sederhana tanpa harus menggunakan komputer mainframe. Sejak saat itu, dengan semakin berkembangnya daya komputasi, telah banyak sekali muncul aplikasi \textit{spreadsheet} baru hingga saat ini.

\textit{Spreadsheet} memiliki beberapa keunggulan dibandingkan dengan aplikasi pengolahan data jenis lain. Keunggulan yang paling terlihat adalah banyak orang yang mengetahui cara penggunaan aplikasi jenis \textit{spreadsheet}. Selain itu, \textit{spreadsheet} jika digunakan dengan benar, memiliki banyak fitur dan kemampuan yang jarang diketahui orang awam. Dengan keunggulan ini, \textit{spreadsheet} sering kali dijadikan pilihan utama dalam pengolahan data.

Bagi orang awam, \textit{spreadsheet} sering terlihat sebagai aplikasi yang digunakan untuk melakukan perhitungan sederhana. Selain itu banyak orang menganggap, penggunaan \textit{spreadsheet} adalah personal sehingga tidak membutuhkan tim atau bantuan orang lain dalam pembuatan sebuah \textit{spreadsheet}. Hal ini tidak dapat dibenarkan, karena jika melihat kasus penggunaannya pada organisasi bisnis yang besar, \textit{spreadsheet} yang dihasilkan sangatlah kompleks dan besar dengan pengembangan yang membutuhkan banyak orang \parencite{Panko1998}.

Kompleksitas dan besarnya ukuran \textit{spreadsheet} inilah yang membuat penggunaan \textit{spreadsheet} pada sebuah bisnis sangatlah rentan akan kesalahan. Sebuah kesalahan kecil dapat berakibat fatal dan memberikan kerugian seperti kehilangan pendapatan, kesalahan pemberian harga, penipuan, dan kegagalan sistem akibat ketergantungan berlebih antar \textit{spreadsheet} \parencite{EUSPRIGAbout}. Telah banyak bukti dan penelitian yang menunjukan bahwa kesalahan pada \textit{spreadsheet} sangat mudah ditemui. Bahkan pada \textit{spreadsheet} yang dibuat dengan sangat hati-hati, masih dapat ditemui sekitar 1 persen atau lebih kesalahan pada formula yang dibuat \parencite{Panko1998}.

Tingginya angka kesalahan yang dapat terjadi pada suatu \textit{spreadsheet} merupakan hal yang sangat krusial terutama didalam bisnis. Metode pencegahan harus dapat dilakukan untuk dapat mengurangi angka kesalahan ini. Beberapa organisasi dapat menerapkan pencegahan dengan cara melakukan tahap-tahap metodologi yakni dengan pembuatan desain awal, melakukan metode \textit{best practice} yang tersedia dan sesuai dengan situasi yang dihadapi, menerapkan \textit{policy} khusus pada saat pembuatan \textit{spreadsheet}, melakukan \textit{testing}, serta pembuatan dokumentasi \parencite{EUSPRIGBestPractice}. Namun, metodologi tersebut masih sangat rentan oleh kesalahan manusia karena perangkat lunak yang digunakan tetap tidak diubah didalam menjalankan metodologi tersebut. 

Untuk dapat mengurangi kesalahan-kesalahan yang sering terjadi pada \textit{spreadsheet} secara lebih mendasar, dibutuhkan bantuan perangkat lunak untuk dapat melakukan kontrol terhadap masukan pengguna. Pada tugas akhir ini akan difokuskan pada pengembangan sebuah aplikasi yang dapat membantu pengguna \textit{spreadsheet} mengurangi kesalahan yang sering terjadi pada pengembangan \textit{spreadsheet}.

\section{Rumusan Masalah}

Kesalahan yang cukup banyak terjadi pada penggunaan \textit{spreadsheet} yakni tidak konsistennya \textit{spreadsheet}. Selain itu sering juga terjadi kesalahan memasukan formula atau penggantian fungsi operasi pada \textit{spreadsheet} secara tidak sengaja. Hal-hal tersebut merupakan beberapa masalah utama yang cukup penting untuk diselesaikan terutama didalam penggunaannya pada bisnis dan komersial. Sehingga, berdasarkan latar belakang yang telah disebutkan pada bagian sebelumnya, akan dibentuk sebuah aplikasi yang membantu mengurangi tingkat kesalahan pada \textit{spreadsheet}. Dalam rangka pemabangunan aplikasi, terdapat beberapa permasalahan yang menjadi perhatian pada tugas akhir ini, yaitu:

\begin{enumerate}
    \item Mengetahui jenis kesalahan yang umum dilakukan pada penggunaan \textit{spreadsheet}
    \item Mengetahui kebutuhan dari perangkat lunak berdasarkan jenis-jenis kesalahan yang sering terjadi
    \item Mengetahui rancangan dari perangkat lunak yang dapat memenuhi kebutuhan tersebut
    \item Mengetahui cara mengimplementasi perangkat lunak yang dibutuhkan
\end{enumerate}

\section{Tujuan}

Tujuan yang ingin dicapai dalam Tugas Akhir ini adalah mengembangkan perangkat lunak yang dapat mengurangi tingkat kesalahan pada \textit{spreadsheet} hal ini dilakukan dengan mengintegrasikan basis data pada penggunaan \textit{spreadsheet}. Dengan adanya perangkat lunak ini, diharapkan dapat memvalidasi data masukan dengan lebih baik. Selain itu, dengan adanya perangkat lunak ini diharapkan dapat meningkatkan efisiensi persebaran \textit{spreadsheet} dengan bantuan basis data yang terintegrasi.

\section{Batasan Masalah}

Untuk mencapai tujuan Tugas Akhir ini, terdapat beberapa batasan-batasan tertentu. Batasan tersebut ditujukan untuk memperjelas dan memfokuskan objek penelitian dan pengembangan tugas akhir. Batasan-batasan masalah pengerjaan tugas akhir adalah sebagai berikut.

\begin{enumerate}
    \item Data yang dapat dijadikan masukan berupa data berbentuk tabel atau formulir sederhana.
    \item Struktur basis data yang digunakan dalam penggunaan aplikasi ini dianggap sudah menemenuhi persyaratan normalisasi dan dapat langsung digunakan. Pembentukan basis data yang sesuai tidak dibahas didalam Tugas Akhir ini.
\end{enumerate}

\section{Metodologi}

Metodologi yang digunakan dalam pengerjaan Tugas Akhir ini yakni:
\begin{enumerate}
    \item Studi Literatur

    Pengerjaan tugas akhir diawali dengan mencari dan mempelajari referensi berupa jurnal ilmiah dan aplikasi-aplikasi yang telah ada sebelumnya yang dapat membantu pengembangan aplikasi yang dibuat pada tugas akhir ini. Literatur yang dicari dan dipelajari berkaitan dengan topik tugas akhir yaitu mengenai \textit{spreadsheet}, penggunaannya pada bisnis, kesalahan yang sering dilakukan dalam pembuatan, metode \textit{quality control} yang dapat dilakukan, serta hal-hal lain yang masih berkaitan dengan topik tugas akhir ini. 
    
    \item Analisis Masalah

    Pada tahap ini dilakukan analisis permasalahan yang berkaitan dengan topik yang diangkat pada tugas akhir ini. Selain itu, dilakukan penentuan spesifikasi dan fitur yang ada pada aplikasi tersebut sebagai bentuk solusi terhadap permasalahan yang dianalisis.

    \item Perancangan Solusi

    Pada tahap ini dilakukan perancangan solusi yang dapat menyelesaikan masalah-masalah yang telah dijelaskan pada bagian analisis masalah. Bagian perancangan ini juga menjelaskan arsitektur yang digunakan untuk membangun perangkat lunak berdasarkan spesifikasi dan metode yang digunakan.

    \item Implementasi

    Pada tahap ini dilakukan pembangunan aplikasi sesuai dengan kebutuhan dan spesifikasi dari hasil analisis masalah serta rancangan solusi yang diajukan.

    \item Pengujian dan Analisis Hasil

    Pada tahap ini dilakukan pengujian dengan menggunakan data set uji yang sesuai dengan batasan masalah ke dalam aplikasi yang diimplementasikan. Selanjutnya dilakukan analisis hasil pengujian dan penarikan kesimpulan.

\end{enumerate}

\section{Sistematika Pembahasan}

Penulisan tugas akhir ini terdiri dari 5 bab, yaitu: BAB I Pendahuluan, BAB II Tinjauan Pustaka, BAB III Analisis dan Perancangan, BAB IV Evaluasi dan Pembahasan, dan BAB V Penutup. 

Bab satu membahas mengenai latar belakang permasalahan, rumusan masalah, tujuan, batasan masalah, metodologi serta sistematika pembahasan yang digunakan. Bab ini juga menjelaskan secara umum isi dari tugas besar serta gambaran dasar dari pelaksanaan tugas akhir.

Bab dua menjelaskan mengenai dasar teori yang digunakan didalam menyelesaikan permasalahan yang diangkat. Teori yang digunakan berasal dari literatur dan referensi yang berhubungan dengan permasalahan yang diangkat seperti hal-hal yang berkaitan dengan \textit{spreadsheet}, penggunaannya pada bisnis, kesalahan yang sering dilakukan dalam pembuatan, serta metode \textit{quality control} yang dapat dilakukan untuk mencegah kesalahan. Dasar teori ini menjadi dasar analisis dan rancangan solusi pada bab selanjutnya.

Bab tiga memaparkan analisa kebutuhan dan permasalahan yang dipilih yakni tingginya tingkat kesalahan yang terjadi pada \textit{spreadsheet}. Dari hasil analisa yang dilakukan, di dapatkan bentuk solusi umum yang dapat digunakan untuk mengatasi permasalahan tersebut. Selanjutnya solusi umum tersebut dibuat rancangan dan arsitekturnya agar dapat diimplementasikan.

Bab empat memperlihatkan hasil pengujian yang dilakukan kepada aplikasi yang dibuat serta analisis dan pembahasan dari pengujian tersebut. Pengujian dilakukan untuk mengetahui keberhasilan aplikasi yang dibuat untuk menyelesaikan permasalahan yang di definisikan pada rumusan masalah.

Bab lima berisikan kesimpulan terhadap hasil implementasi dan solusi yang dipaparkan untuk menyelesaikan permasalahan. Selain itu, terdapat bagian saran yang memaparkan saran pengembangan dan perbaikan yang dapat dilakukan untuk memperkaya fitur dan menyelesaikan permasalahan yang lebih luas.

\section{Jadwal Pelaksanaan}
\newsavebox\mybox
\begin{lrbox}{\mybox}
    \begin{ganttchart}[
    vgrid={*{6}{draw=none}, dotted},
    x unit=.05cm,
    y unit title=.6cm,
    y unit chart=.6cm,
    time slot format=isodate,
    time slot format/start date=2016-09-01]{2016-09-01}{2017-04-30}
    \ganttset{bar height=.6}
    \gantttitlecalendar{year, month} \\
    \ganttbar[bar/.append style={fill=blue}]{Tahap 1}{2016-09-01}{2016-11-30}\\
    \ganttbar[bar/.append style={fill=blue}]{Tahap 2}{2016-10-01}{2016-11-15}\\
    \ganttbar[bar/.append style={fill=blue}]{Tahap 3}{2016-11-01}{2016-12-15}\\
    \ganttbar[bar/.append style={fill=blue}]{Tahap 4}{2016-12-15}{2017-03-01}\\
    \ganttbar[bar/.append style={fill=blue}]{Tahap 5}{2017-02-01}{2017-04-30}
    \end{ganttchart}
\end{lrbox}

Pengerjaan tugas akhir ini direncanakan mulai pada September 2016 sampai April 2016. Pelaksanaan tugas akhir ini dibagi menjadi 5 tahap yang dapat dipetakan kepada metodologi pengerjaan sebagai berikut,
\begin{enumerate}
    \item Tahap 1: Studi Literatur
    \item Tahap 2: Analisis Masalah
    \item Tahap 3: Perancangan Solusi
    \item Tahap 4: Implementasi
    \item Tahap 5: Pengujian dan Analisis Hasil
\end{enumerate}
Jadwal pelaksanaan tugas akhir berdasarkan metodologi pengerjaan tugas akhir dapat dilihat pada Tabel \ref{Gantt-Chart} dibawah ini.
\begin{table}[htb]
\centering
\caption{Gantt Chart jadwal pelaksanaan tugas akhir}
\label{Gantt-Chart}
\tikz{
  \node[inner sep=0pt,outer sep=0pt] (gantt)
  {\begin{tabular}{c}
    \toprule
    \resizebox{\textwidth}{!}{\usebox\mybox} \\
    \bottomrule
   \end{tabular}%
   };
}   
\end{table}