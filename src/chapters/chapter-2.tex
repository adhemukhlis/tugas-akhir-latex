\chapter{Studi Literatur}

Pada bab ini akan mendeskripsikan kajian literatur yang terkait dengan persoalan tugas akhir. Studi literatur ini akan dijadikan dasar dalam melakukan penyelesaian persoalan.

\section{Penggunaan \textit{Spreadsheet}}
Secara harafiah, \textit{spreadsheet} adalah suatu perangkat lunak yang dapat melakukan kalkulasi terhadap angka serta mengorganisir informasi yang ada di dalamnya berdasarkan kolom dan baris \parencite{meriamwebster-spreadsheet}. \textit{Spreadsheet} dapat digunakan untuk melakukan kalkulasi terhadap suatu rumus atau formula yang sulit jika dikalkulasikan dengan cara manual. Selain itu, \textit{spreadsheet} dapat juga digunakan untuk melakukan ramalan terhadap suatu perubahan variabel masukan. Pada perkembangannya, \textit{spreadsheet} memiliki fitur-fitur tambahan seperti visualisasi data dan ekstraksi data penting dari kumpulan data yang ada.

Penelitian tentang penggunaan \textit{spreadsheet} pada bisnis pernah dilakukan sebelumnya pada tahun 2014. Subjek yang diteliti adalah akuntan manajemen \parencite{Bradbard2014}. Pada penelitian tersebut, didapatkan gambaran umum mengenai penggunaan \textit{spreadsheet} secara umum. Menurut hasil penelitian tersebut beberapa fitur yang sering digunakan oleh pengguna \textit{spreadsheet} secara terurut dari yang paling sering digunakan adalah sebagai berikut,

\begin{enumerate}
    \item Menghitung fungsi matematika dasar (tambah, kurang, kali, bagi, dan lainnya)
    \item Mengelola \textit{worksheet} dan \textit{workbook} (menambahkan, menghapus, merubah nama, dan lainnya)
    \item Melakukan perubahan format dasar (menebalkan, memberi garis bawah, format angkat, dan lainnya)
    \item Melakukan pengurutan data, penghitungan subtotal, serta meringkas data
    \item Menggunakan fitur \textit{cell addressing} baik absolut maupun relatif
    \item Penggunaan fungsi kondisi (IF, COUNTIF), fungsi logika (AND, OR), fungsi pencarian (VLOOKUP, HLOOKUP), menautkan \textit{workbook} lain, serta fungsi pembulatan (ROUND, CEILING, FLOOR)
\end{enumerate}

Penggunaan \textit{spreadsheet} sangat bergantung kepada domain bisnis atau organisasi yang menggunakan. Pada bisnis yang berorientasi komersial, \textit{spreadsheet} dapat digunakan sebagai alat bantu perhitungan laba, pengeluaran, investasi, dan pajak. Pada organisasi-organisasi non komersial, \textit{spreadsheet} dapat digunakan sebagai salah satu bentuk basis data yang menangani penyimpanan, pengelolaan, dan pengumpulan data yang mudah dan cepat.

\section{Kesalahan dalam Penggunaan \textit{Spreadsheet}}
Jelasin banyaknya kesalahan pada penggunaan spreadsheet!!!

\section{Tipe Kesalahan dalam Penggunaan \textit{Spreadsheet}}
Tingkat fleksibilitas \textit{spreadsheet} yang tinggi memberikan keleluasaan kepada penggunanya untuk melakukan banyak manipulasi dan pengelolaan data. Tingginya fleksibilitas ini dapat berakibat mudahnya \textit{human error} terjadi pada saat penggunaan \textit{spreadsheet} yang menyebabkan terjadinya kesalahan-kesalahan pada data. Tipe-tipe kesalahan pada \textit{spreadsheet} dapat dibagi menjadi dua jenis tipe kesalahan yakni kesalahan kuantitatif, dan kesalahan kualitatif \parencite{Panko1998}.

    \subsection{Kesalahan Kualitatif}
    Kesalahan kualitatif merupakan kesalahan yang berhubungan dengan kualitas \textit{spreadsheet} tersebut. Beberapa kesalahan yang dapat diklasifikasikan sebagai kesalahan kualitatif adalah \parencite{Powell2009}:

    \begin{enumerate}
        \item Melakukan \textit{hard-code} pada suatu angka di dalam formula
        \item Menggunakan formula yang panjang dalam perhitungan
        \item Susunan data yang tidak direncanakan dengan baik
        \item Tidak adanya dokumentasi mengenai \textit{spreadsheet} yang dibuat
    \end{enumerate}

    Kesalahan ini tidak langsung mengakibatkan nilai hasil keluaran yang salah namun menurunkan kualitas dari \textit{spreadsheet} tersebut \parencite{Rajalingham2001}. Selain itu, kesalahan kualitatif dapat menyebabkan kesalahan kuantitatif terutama pada saat penggunaan fungsi analisis \textit{what-if} pada \textit{spreadsheet} \parencite{Panko1998}.

    \subsection{Kesalahan Kuantitatif}
    Kesalahan ini mengakibatkan \textit{spreadsheet} mengeluarkan hasil dan nilai yang salah didalam operasi perhitungannya. Kesalahan kuantitatif dapat dibagi menjadi tiga tipe kesalahan yakni \parencite{Panko1998}:

    \begin{enumerate}
        \item Kesalahan mekanikal (\textit{mechanical error}) yang biasanya terjadi akibat kesalahan pengetikan angka atau rujukan sel yang salah pada suatu formula
        \item Kesalahan logika (\textit{logical error}) yang terjadi pada pembuatan formula yang salah atau penggunaan fungsi yang tidak tepat
        \item Kesalahan akibat kelalaian pada interpretasi situasi atau spesifikasi yang diberikan sehingga \textit{spreadsheet} yang dihasilkan tidak sesuai dengan domain permasalahan yang ada \parencite{Powell2009} (\textit{ommision error})
    \end{enumerate}

\section{Penanganan Kesalahan pada \textit{Spreadsheet}}
Berdasarkan penelitian yang dilakukan oleh Panko \parencite{Panko1998}, dijabarkan beberapa metode untuk menangani dan mengurangi kesalahan yang sering terjadi. Beberapa metode yang dapat digunakan yakni:

    \begin{enumerate}
        \item Membangun \textit{preliminary design} sebelum pembuatan \textit{spreadsheet} agar terdapat perencanaan yang baik di dalam pembangunan data di dalam \textit{spreadsheet}
        \item Melakukan proteksi terhadap sel yang tidak boleh diubah.
        \item Melakukan pengecekan terhadap semua rumus dan formula yang dimasukan bahkan hingga rumus yang cukup sederhana dengan cara melakukan pengecekan manual.
        \item Membuat dokumentasi untuk \textit{spreadsheet} yang dibuat.
        \item Tidak menekan pembuat \textit{spreadsheet} terhadap kesalahan yang dibuat dengan memberikan hukuman. Kesalahan yang terjadi pada \textit{spreadsheet} umumnya masih berada pada batas normal \textit{human error} sehingga memberikan hukuman akan membuat rasa takut dalam melaporkan kesalahan.
        \item Melakukan inspeksi terhadap formula, rumus, dan kode yang dibuat baik oleh individual maupun secara berkelompok.
    \end{enumerate}

\section{Basis Data Relasional}
Perujukan literatur \parencite{knuth2008art} dapat dilakukan dengan menambahkan entri baru di berkas. Tulisan ini merujuk pada \parencite{knuth2001art}

    \subsection{Subbab}

    \blindtext

    \begin{figure}[h]
        \centering
        \includegraphics[width=0.8\textwidth]{resources/chapter-2-infrastructure-diagram.png}
        \caption{Contoh gambar}
    \end{figure}

    \subsubsection{Subsubbab}

    \blindtext

\section{Struktur \textit{File} pada \textit{Spreadsheet}}
Perujukan literatur \parencite{knuth2008art} dapat dilakukan dengan menambahkan entri baru di berkas. Tulisan ini merujuk pada \parencite{knuth2001art}

    \subsection{Subbab}

    \blindtext

    \begin{figure}[h]
        \centering
        \includegraphics[width=0.8\textwidth]{resources/chapter-2-infrastructure-diagram.png}
        \caption{Contoh gambar}
    \end{figure}

    \subsubsection{Subsubbab}

    \blindtext

\section{Ekstraksi Data}
Perujukan literatur \parencite{knuth2008art} dapat dilakukan dengan menambahkan entri baru di berkas. Tulisan ini merujuk pada \parencite{knuth2001art}

    \subsection{Subbab}

    \blindtext

    \begin{figure}[h]
        \centering
        \includegraphics[width=0.8\textwidth]{resources/chapter-2-infrastructure-diagram.png}
        \caption{Contoh gambar}
    \end{figure}

    \subsubsection{Subsubbab}

    \blindtext

\section{??Aplikasi Formulir Berbasis Web}
Perujukan literatur \parencite{knuth2008art} dapat dilakukan dengan menambahkan entri baru di berkas. Tulisan ini merujuk pada \parencite{knuth2001art}

    \subsection{Subbab}

    \blindtext

    \begin{figure}[h]
        \centering
        \includegraphics[width=0.8\textwidth]{resources/chapter-2-infrastructure-diagram.png}
        \caption{Contoh gambar}
    \end{figure}

    \subsubsection{Subsubbab}

    \blindtext

\section{Studi Terkait}
\blindtext
