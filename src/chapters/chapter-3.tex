\chapter{Analisis Masalah Penangganan Kesalahan Pada \textit{Spreadsheet}}
Pada bab ini akan diuraikan analisis persoalan penangganan kesalahan pada \textit{spreadsheet} yang telah diuraikan pada Bab I. Hasil dari bab ini digunakan untuk merancang aplikasi yang akan diimplementasikan seperti yang dijelaskan pada Bab IV.

\section{Model Interaksi Pengguna}
Di dalam pembangunan perangkat lunak \textit{spreadsheet} untuk mengurangi kesalahan, dapat diidentifikasikan dua model interaksi yang dapat diimplementasi. Model interaksi yang pertama adalah menggunakan formulir sebagai basis masukan data dan model yang kedua adalah menggunakan aplikasi \textit{spreadsheet} secara langsung sebagai media input data.

\subsection{Berbasis Formulir}


\subsection{Berbasis \textit{Spreadsheet}}

\section{}

\section{Penentuan Bagian Data dan Label}
\subsection{Manual oleh Pengguna}

\subsection{Secara Otomatis}
Pada penggunaan \textit{spreadsheet}, terutama pada tipe \textit{data frame}, terdapat dua jenis bagian inti yang selalu ada pada \textit{spreadsheet} yakni bagian label dan data. Label merupakan penjelasan dari suatu data. Seperti yang telah dijelaskan pada Subbab \ref{Metode Pencarian}, mekanisme untuk mengidentifikasi label dan data dapat dilakukan melalui 3 tahapan yakni, \textit{frame finder}, \texit{hierarchy extractor}, dan \textit{tuple builder}.

\section{Identifikasi Domain Data Masukan Pengguna}
\subsection{Ditentukan oleh Sistem}

\subsection{Ditentukan oleh Pengguna}


\section{Auto-commit atau Commit-push type?}

\section{Pengecekan Integrasi dan Kesesuaian Data}



\section{Penyimpanan dan Pemulihan Data}



====




\section{Analisis Permasalahan Pada \textit{Spreadsheet}}
\blindtext
Masukin yang versioning, data validatoin (type validation: int, string, date dll), database alternative, make use of DB ACID
http://www.mrc-productivity.com/blog/2012/11/spreadsheet-misuse-why-it-happens-and-how-to-stop-it/
http://www.burns-stat.com/documents/tutorials/spreadsheet-addiction/
Lebih cocok menangani permasalahan misuse spreadsheet sebagai basis data


\section{ Umum}
\blindtext

\section{Rancangan Solusi}
\blindtext